\documentclass[10pt,landscape,a4paper,twocolumn]{article}

\setlength{\columnsep}{20pt}

\usepackage[left=1.0cm, right=1.0cm, top=1.5cm, bottom=1.0cm, headsep=0.4cm]{geometry}
\usepackage{amsmath}
\usepackage{amssymb}
\usepackage{fontspec}
\usepackage{kotex}
\usepackage{graphicx}
\usepackage{setspace}
\usepackage{listings}
\usepackage{comment}
\usepackage{import}
\usepackage{wrapfig}
\usepackage{url}
\usepackage{array}
\usepackage[normal]{engord}
\usepackage[svgnames,table]{xcolor}

\usepackage{fancyhdr}
\pagestyle{fancy}
\fancyhead[R]{\thepage}
\fancyhead[L]{Seoul National University - PLEASE OPEN TESTDATA}

\setmonofont[
    BoldFont = consolab.ttf,
    ItalicFont = consolai.ttf
]{consola.ttf}
\setlength\parindent{0pt}
\usepackage[parfill]{parskip}

\definecolor{dkgrey}{RGB}{127, 127, 127}

\lstset{
    basicstyle=\footnotesize\ttfamily,
    breaklines=true,
    breakindent=1.1em
%    numbers=left,
%    numberstyle=\footnotesize\ttfamily\color{dkgrey},
%    numbersep=5pt
%    frame=trbl
}

\lstdefinestyle{mycpp}{
  language=[GNU]C++,
  keywordstyle=\color{blue},
  commentstyle=\itshape\color{purple!40!black},
  stringstyle=\color{orange},
}

\begin{document}
\tableofcontents

\section{Basic Template}

\subsection{C++}
\lstinputlisting[style=mycpp]{../Templates/base/template.cpp}

\subsection{Python}
\lstinputlisting[language=Python]{../Templates/base/template.py}

\section{Data Structure}

\subsection{Segment Tree}
\lstinputlisting[style=mycpp]{../Templates/datastructures/segtree_base.cpp}

\subsection{Lazy Propagation}
\lstinputlisting[style=mycpp]{../Templates/datastructures/lazy_propagation.cpp}

\subsection{Merge Sort Tree}
\lstinputlisting[style=mycpp]{../Templates/datastructures/merge_sort_tree.cpp}

\subsection{SQRT Decomposition}
\lstinputlisting[style=mycpp]{../Templates/datastructures/sqrt_decomposition.cpp}

\section{Graph}

\subsection{Dijkstra}
\lstinputlisting[style=mycpp]{../Templates/graph/dijkstra.cpp}

\subsection{Kosaraju}
\lstinputlisting[style=mycpp]{../Templates/graph/kosaraju.cpp}

\subsection{Floyd Warshall}
\lstinputlisting[style=mycpp]{../Templates/graph/floyd_warshall.cpp}

\subsection{Centroid Decomposition}
\lstinputlisting[style=mycpp]{../Templates/graph/tree/centroid_decomposition.cpp}

\subsection{Heavy Light Decomposition}
\lstinputlisting[style=mycpp]{../Templates/graph/tree/heavy_light_decomposition.cpp}

\subsection{LCA}
\lstinputlisting[style=mycpp]{../Templates/graph/tree/LCA.cpp}

\section{Math}

\subsection{Combination}
\lstinputlisting[style=mycpp]{../Templates/combinatorics/combination.cpp}

\subsection{Linear Sieve}
\lstinputlisting[style=mycpp]{../Templates/number/linear_sieve.cpp}

\section{String}

\subsection{KMP}
\lstinputlisting[style=mycpp]{../Templates/string/kmp.cpp}

\subsection{Manacher}
\lstinputlisting[style=mycpp]{../Templates/string/manacher.cpp}

\section{Etc}

\subsection{Convex Hull Trick}
\lstinputlisting[style=mycpp]{../Templates/dp/convex_hull_trick.cpp}

\subsection{Union Find}
\lstinputlisting[style=mycpp]{../Templates/utils/union_find.cpp}

\subsection{mo's algorithm}
\lstinputlisting[style=mycpp]{../Templates/utils/mos.cpp}

\section{체계적인 접근을 위한 질문들}

``알고리즘 문제 해결 전략'' 에서 발췌함
\begin{itemize}
  \item 비슷한 문제를 풀어본 적이 있던가?
  \item 단순한 방법에서 시작할 수 있을까? (brute force)
  \item 내가 문제를 푸는 과정을 수식화할 수 있을까? (예제를 직접 해결해보면서)
  \item 문제를 단순화할 수 없을까?
  \item 그림으로 그려볼 수 있을까?
  \item 수식으로 표현할 수 있을까?
  \item 문제를 분해할 수 있을까?
  \item 뒤에서부터 생각해서 문제를 풀 수 있을까?
  \item 순서를 강제할 수 있을까?
  \item 특정 형태의 답만을 고려할 수 있을까? (정규화)
\end{itemize}

\end{document}
